% -*- coding: utf-8 -*-
%% Небольшой хак для кривого hyperref. Чуть лучше гиперссылка на эту главу
%\chapter*{К~читателю}\phantomlabel{chapter:to_the_reader}%
%\addcontentsline{toc}{chapter}{К~читателю}
\chapter{К~читателю}\label{chapter:to_the_reader}

\initial{0.25ex}{0.0ex}{Н}{\kern0.65ex есмотря на то,} что литературы на
тему Лиспа достаточно и она легкодоступна, эта книга имеет свою нишу.
Программирование требует понимания фундаментальных основ языка; для Лиспа и
Scheme ими являются нетривиальные вещи вроде функций высшего порядка, объектов,
продолжений и тому подобного. Их незнание и непонимание преграждает вам путь
в~будущее, так как то, что ещё сегодня считалось сложным, завтра уже станет
нормой для образованного человека.

Для объяснения природы данных сущностей, их происхождения и разновидностей
нам придётся серьёзно углубиться в~детали. Ходит поговорка, что лисперы знают
ценность всего, но не~ведают цены. Данная книга также направлена на сокращение
этой пропасти в~понимании языка с~помощью детального изучения его семантики,
а~также реализации различных возможностей Лиспа, которые были изобретены за его
более чем тридцатилетнюю историю.

Лисп "--- это приятный язык, на котором многие фундаментальные и нетривиальные
вещи выражаются простым образом. Вместе с~ML, своим строго типизированным
собратом, (почти) лишённым побочных эффектов, он является типичным
представителем семейства аппликативных языков программирования. Изучение
концепций, на которых это семейство основано, без сомнения будет полезно для
студентов и учёных-информатиков наших и будущих лет. Основанные на идее
\term{функции}, "--- идее, которая веками оттачивалась и уточнялась математикой,
"--- аппликативные языки присутствуют практически везде, где присутствуют
вычисления, проявляясь в~различных формах: начиная перенаправлением потоков
в~\UNIX, заканчивая языком расширений редактора~Emacs и многими другими
скриптовыми языками. Использование таких инструментов без понимания основного их
механизма "--- комбинации "--- подобно попыткам выразить мысль с~помощью
отдельных слов вместо цельного предложения. Для выживания может быть достаточно
нескольких заученных фраз, но для полноценной жизни требуется вся мощь языка.


\section*{Аудитория}\label{pref/sect:audience}

Книга предназначена для широкой аудитории специалистов:
\begin{itemize}
  \item для выпускников вузов и студентов, которые изучают приёмы
        реализации языков программирования; аппликативных или нет,
        интерпретацию или компиляцию "--- не~важно;

  \item для программистов на Лиспе или Scheme, желающих чётче понимать
        нюансы и стоимость используемых ими конструкций, дабы писать
        более эффективные и переносимые программы;

  \item для всех любителей аппликативных языков, которые найдут в~этой
        книге множество интересных размышлений на свою любимую тему.
\end{itemize}


\section*{Философия}\label{pref/sect:philosophy}

Данная книга основана на курсе лекций, читаемом в~магистратуре Университета
Пьера и Марии~Кюри; некоторые части курса также преподаются в~Политехнической
школе.

Темы, рассматриваемые здесь, обычно следуют за вводным курсом аппликативных
языков вроде Лиспа, Scheme или~ML, так как подобные курсы чаще всего
заканчиваются детальным разбором рассматриваемого языка. Цель этой книги "---
как можно шире покрыть тему семантики аппликативных языков и разработки их
интерпретаторов и компиляторов. Здесь приведено двенадцать интерпретаторов и
два компилятора (в~байт-код и в~Си). Не~обходится стороной и
объектно"=ориентированная модель (рассматриваемая на примере \Meroon). Также,
в~отличие от многих других книг, эта не~пренебрегает такими существенными для
семейства Лиспов вещами как рефлексия, интроспекция, динамическая кодогенерация
и, конечно~же, макросы.

Отчасти эта книга вдохновлена двумя работами: <<\english{Anatomy of~Lisp}>>
\cite{all78}, рассматривающей подходы к~реализации Лиспа в~семидесятых годах,
и~<<\english{Operating System Design: The XINU Approach}>> \cite{com84}, где
приводится весь необходимый код без сокрытия любых деталей работы операционной
системы, что полностью убеждает читателя в~верности изложения.

В~таком~же духе "--- точности, а не~лаконичности "--- написана и эта книга,
главным вопросом которой есть семантика аппликативных языков в~общем и Scheme
в~частности. Исследуя множество реализаций, рассматривая их различные аспекты,
мы узнаем с~максимальной точностью, как строится любая подобная система. Мы
рассмотрим большую часть проблемных вопросов, вызывающих расколы в~сообществе;
каждая из этих проблем будет изучена, варианты её решения "--- реализованы,
сравнены и проанализированы. <<Я~расскажу всё, что знаю>>, чтобы вы, читатель,
никогда не~зашли в~тупик из"~за недостатка информации, и, более того, имея
такой фундамент знаний, могли самостоятельно экспериментировать
с~рассматриваемыми концепциями.

Именно поэтому вы можете получить в~электронном виде полный код всех программ,
приведённых в~этой книге (подробности на странице~\pageref{pref/sect:source}).


\section*{Структура}\label{pref/sect:structure}

Книга разделена на две части. Первая часть начинается реализацией наивного
интерпретатора Лиспа и рассматривает в~основном семантику Scheme. Здесь нам
требуется точность повествования, поэтому мы будем раз за разом уточнять и
переопределять различными способами пространства имён (\Lisp1, \Lisp2 и~т.\,д.),
продолжения (и~связанные с~ними управляющие конструкции), присваивание и
изменяемые структуры данных. Мы заметим, что по мере того, как определяемый
язык обрастает возможностями, его определение становится всё более простым,
приближаясь к~$\lambda$"=исчислению. Полученное таким образом описание языка мы
превратим в~его денотационный, строго математический эквивалент.

Более шести лет практики преподавания убедили меня в~том, что именно такой
подход постепенного уточнения языка необходим для мягкого знакомства с~темой
исследования языков вообще и денотационной семантикой вычислений в~частности
"--- темой, которую мы не~можем себе позволить обойти стороной.

Вторая часть книги следует иным путём. Преследуя цель сделать наивную
реализацию денотационного интерпретатора более эффективной, мы коснёмся темы
ускорения интерпретации (заранее вычисляя неизменные величины), а потом
реализуем эту предварительную обработку (с~помощью прекомпиляции) для нашего
компилятора в~байт-код. В~этой части подготовка программы к~исполнению и
собственно исполнение чётко отделены, поэтому здесь будут рассматриваться такие
темы как динамические вычисления (\ic{eval}), рефлексия (окружения как объекты
первого класса, самоинтерпретация, <<башня>> интерпретаторов), семантика
макросов. Далее мы реализуем транслятор Scheme в~код на языке~Си.

Завершается книга реализацией объектно"=ориентированной системы, которая
существенно поможет нам в~реализации некоторых интерпретаторов и компиляторов.

Как известно, повторенье "--- мать ученья. Все приведённые интерпретаторы
намеренно написаны в~различных стилях: наивном, объектно"=ориентированном,
основанном на замыканиях, денотационном и~т.\,д. Это позволит рассмотреть
множество приёмов, используемых при реализации аппликативных языков. Также это
подтолкнёт вас на размышления о~различиях между ними. Понимание этих различий
(см.~таблицу~\ref{pref/table:signatures} с~подсказками) является истинным
пониманием языка и его реализаций. Лисп "--- это не~одна из таких реализаций,
это \emph{семейство} диалектов, каждый из которых имеет свой уникальный набор
черт, которые мы будем рассматривать.

\begin{table}
\begin{center}\begin{tabular}{cl}
Глава & Прототип                    \\
\hline
1  & \ic{(eval exp env)}            \\
2  & \ic{(eval exp env fenv)}       \\
   & \ic{(eval exp env fenv denv)}  \\
   & \ic{(eval exp env denv)}       \\
3  & \ic{(eval exp env cont)}       \\
4  & \ic{(eval e r s k)}            \\
5  & \ic{((meaning e) r s k)}       \\
6  & \ic{((meaning e r) sr k)}      \\
   & \ic{((meaning e r tail?) k)}   \\
   & \ic{((meaning e r tail?))}     \\
7  & \ic{(run (meaning e r tail?))} \\
10 & \ic{(->C (meaning e r))}
\end{tabular}\end{center}
\caption{Прототипы интерпретаторов и компиляторов.}
\label{pref/table:signatures}
\end{table}

Главы более-менее независимы, занимают примерно по~40~страниц; каждая глава
имеет список упражнений, ответы к~которым можно найти в~конце книги. Список
литературы содержит не~только исторически важные книги, позволяющие отследить
развитие Лиспа с~1960~года, но и современные труды.


\section*{Предварительные знания}\label{pref/sect:prereqs}

Хоть я и надеюсь, что книга будет увлекательной и содержательной, но она
не~обязательно будет лёгкой для чтения. Некоторые описанные здесь вещи можно
постичь, только если прикладывать усилия, соответствующие их сложности. Говоря
языком куртуазных романов, некоторые предметы воздыханий открывают свою истинную
красоту и обаяние только тогда, когда мы учтиво, но непреклонно штурмуем их;
если их богатый и непростой внутренний мир не~будет под постоянной осадой, они
так и останутся неприступными.

Изучение сущности языков программирования требует владения инструментами вроде
$\lambda$"=исчисления и денотационной семантики. Хотя повествование и будет
мягко, последовательно и логично переходить от одной темы к~следующей, это
не~сможет избавить вас ото всех необходимых усилий.

Вам потребуются некоторые предварительные знания о~Лиспе или Scheme;
в~частности, знание примерно тридцати базовых функций и умение понимать рекурсию
без чрезмерного умственного напряжения. Основным языком этой книги выбран Scheme
(его краткий обзор можно найти на странице~\pageref{pref/sect:scheme-summary}),
а также его объектно"=ориентированное расширение \Meroon. Данное расширение
поможет нам в~рассмотрении некоторых проблем представления и реализации структур
данных.

Все приведённые в~книге программы были протестированы и действительно работают
в~интерпретаторе Scheme. А~для тех, кто усвоит материал этой книги, не~будет
составлять особого труда портировать их куда~угодно!


\section*{Благодарности}\label{pref/sect:thanks}

Я~должен поблагодарить организации, которые обеспечили меня оборудованием
(Apple~Mac~SE/30, затем Sony~NEWS~3260, впоследствии разнообразными PC и
PowerBook) и~вообще сделали эту книгу возможной: Политехническую школу,
Государственный институт исследований в~области информатики и~автоматики
(INRIA\kern-0.1em), Национальный центр научных исследований (CNRS).

Также я хотел~бы поблагодарить тех, кто помогал мне всем, чем мог, в~создании
этой книги. В~особом долгу я перед Софи~Англад, Жози~Бирон, Кэтлин~Коллэвей,
Жеромом~Шейёксом, Жаном-Мари~Жеффруа, Кристианом~Жюльеном, Жан-Жаком~Лакрампом,
Мишелем~Леметром, Люком~Моро, Жаном-Франсуа~Перро, Дэниелом~Риббенсом,
Бернардом~Серпеттом, Мануэлем~Серрано, Пьером~Ве, а также перед моей музой,
Клэр~Н.{\fnstyle{\RaggedRight}\trnote*{Sophie Anglade, Josy Byron, Kathleen
Callaway, J\'er\^ome Chaillox, Jean-Marie Geffroy, Christian Jullien,
Jean-Jacques Lacrampe, Michel Lema{\^\i}tre, Luc Moreau, Jean-Fran\c cois
Perrot, Daniel Ribbens, Bernard Serpette, Manuel Serrano, Pierre Weis,
Claire~N. "--- \emph{Прим.~перев.}}}

Конечно~же, все ошибки, которые, к~сожалению, неизбежно присутствуют в~тексте,
являются моими собственными.


\section*{Нотация}\label{pref/sect:notation}

Фрагменты программ будут набраны \textcd{таким шрифтом, который несомненно
напомнит вам о~старых добрых печатных машинках}. Некоторые слова в коде также
будут набраны \textit{курсивом} для обозначения понятий, подразумеваемых на
месте этих слов.

\indexC*{.->}{\protect\is}
\indexC*{.=}{\protect\equals}
Знак {\is} читается: <<имеет значение>>, а знак {\equals} обозначает
эквивалентность, <<имеет то~же значение, что~и>>. При разборе вычисления
выражений после вертикальной черты мы будем записывать окружение, в~котором
проводятся вычисления. Вот пример, иллюстрирующий эти соглашения:

\begin{code:lisp}
(let ((a (+ b 1)))
  (let ((f (lambda () a)))
    (foo (f) a) ) )|\begin{where}
                    \- b {\is} 3
                    \- foo {\eq} cons
                    \end{where}|

|\eq| (let ((f (lambda () a))) (foo (f) a))|\begin{where}
                                            \- a {\is} 4
                                            \- b {\is} 3
                                            \- foo {\eq} cons
                                            \- f {\eq} (lambda () a)\begin{where}
                                                                   \- a {\is} 4
                                                                   \end{where}
                                            \end{where}|
|\eq| (foo (f) a)|\begin{where}
                  \- a {\is} 4
                  \- b {\is} 3
                  \- foo {\eq} cons
                  \- f {\eq} (lambda () a)\begin{where}
                                         \- a {\is} 4
                                         \end{where}
                  \end{where}|
|\is| (4 . 4)
\end{code:lisp}

Все имена переменных и сообщения об~ошибках в~приводимых программах мы будем
записывать на английском "--- <<родном языке>> Scheme.

Мы будем использовать несколько нестандартных функций вроде \ic{gensym}, которая
генерирует символы, гарантированно не~встречавшиеся ранее в~тексте программы.
В~десятой главе также будут применяться функции \ic{format} и \ic{pp} для
форматированного вывода (pretty-printing). Эти функции есть в~большинстве
реализаций Лиспа и~Scheme.

Некоторые выражения имеют смысл только для какого"~то из диалектов Лиспа вроде
\CommonLisp, Dylan, \EuLisp, \ISLisp, \LeLisp,\footnote*{{\LeLisp} является
торговой маркой INRIA.} Scheme и~т.\,д. В~этом случае мы будем писать рядом
название диалекта:

\begin{code:lisp}
(defdynamic fooncall          |\dialect{\ISLisp}|
  (lambda (one :rest others)
    (funcall one others) ) )
\end{code:lisp}

Дабы было легче ориентироваться в~этой книге, мы будем использовать
обозначение {\seePage*} для перекрёстных ссылок на~страницы. Похожая нотация
будет использоваться при необходимости указать на упражнение: {\seeEx*}. Также
в~книге есть предметный указатель со~ссылками на все определяемые функции.
\seePage[chapter:index]


\section*{Краткий обзор Scheme}\label{pref/sect:scheme-summary}

Для изучения Scheme существует множество отличных книг, вроде \cite{as85},
\cite{dyb87}, \cite{sf89}. Мы~же будем опираться на спецификацию, описанную
в~документе <<Revised revised revised revised revised Report on Scheme>>,
название которого часто сокращают до \RnRS.~\cite{kcr98}

Сейчас мы лишь набросаем основные характерные черты этого диалекта; те черты,
которые потом будут подробно проанализированы по мере улучшения понимания языка.

В~Scheme можно использовать символы, знаки,\trnote*{Если возможны разночтения,
то слово \term{знак} будет использоваться в~смысле <<печатный символ>>
(character), а слово \term{символ} "--- в~привычном для Лиспа значении
(symbol). "--- \emph{Прим.~перев.}} строки, списки, числа, логические значения,
векторы, порты и~функции (или процедуры, как их принято называть в~Scheme).

Каждый из этих типов данных имеет соответствующий предикат: \ic{symbol?},
\ic{char?}, \ic{string?}, \ic{pair?}, \ic{number?}, \ic{boolean?}, \ic{vector?},
\ic{procedure?}.

Помимо них в~наличии есть процедуры-аксессоры и модификаторы для тех типов,
где это имеет смысл: \ic{string-ref}, \ic{string-set!}, \ic{vector-ref}
и~\ic{vector-set!}.

Для списков они называются \ic{car}, \ic{cdr}, \ic{set-car!} и~\ic{set-cdr!}.

Функции \ic{car} и \ic{cdr} могут комбинироваться. Например, для доступа
ко~второму элементу списка используется \ic{cadr}.

Все значения этих типов могут быть непосредственно записаны в~программе.
С~символами и числами всё очевидно. Перед знаками пишется префикс
\ic{\#\bslash}, например: \ic{\#\bslash Z}, \ic{\#\bslash +},
\ic{\#\bslash space}. Строки окружаются \ic{"}кавычками\ic{"}, списки "---
\ic{(}круглыми скобками\ic{)}. Логические значения записываются как \ic{\#t}
и \ic{\#f} соответственно. Для записи векторов используется синтаксис
\ic{\#(do~re~mi)}. Естественно, такие значения могут быть построены и
динамически с~помощью \ic{cons}, \ic{list}, \ic{string}, \ic{make-string},
\ic{vector}, \ic{make-vector}. Также в~наличии есть функции приведения типов
вроде \ic{string->symbol} и~\ic{int->char}.

Ввод-вывод обеспечивают следующие функции: \ic{read} читает вводимые выражения,
\ic{display} выводит их на экран, а \ic{newline} переходит на следующую строку.

\bigskip

\indexR{форма!концепция Scheme}
Программы на~Scheme представляются так называемыми \term{формами}.

\indexC{begin}
Форма~\ic{begin} позволяет сгруппировать формы и вычислить их последовательно;
например, \ic{(begin (display~1) (display~2) (newline))}.

\indexC{if}\indexC{cond}\indexC{else}
\indexE{Scheme!логические значения}
\indexR{логические значения!в~Scheme}
Есть несколько форм ветвления. Простейшей из них является \ii{if--then--else},
которая на~Scheme так и записывается: \ic{(if \ii{условие} \ii{тогда}
\ii{иначе})}. Если вариантов больше двух, то для этого случая в~Scheme есть
формы \ic{cond} и~\ic{case}. Форма~\ic{cond} содержит список утверждений, каждое
из которых начинается с~условия "--- выражения, возвращающего логическое
значение, "--- за~которым располагается последовательность других форм
(следствие). Она последовательно вычисляет условия утверждений до тех пор,
пока одно из них не~вернёт истину (а~точнее: не~ложь, не~\ic{\#f}); затем
вычисляется следствие данного утверждения, и результат его вычисления становится
результатом всей формы \ic{cond}. Вот пример использования этой формы, который
заодно показывает ключевое слово~\ic{else}:

\begin{code:lisp}
(cond ((eq? x 'flip) 'flop)
      ((eq? x 'flop) 'flip)
      (else (list x "neither flip nor flop")) )
\end{code:lisp}

\indexC{case}
Форма~\ic{case} похожа на~\ic{cond}, но она принимает первым параметром форму,
на основе значения которой производится выбор между вариантами. Каждый из
вариантов в~начале содержит список значений, которые подходят для него. Как
только найден подходящий вариант, он вычисляется и этот результат становится
результатом всей формы~\ic{case}. Аналогично, в~конце может стоять универсальный
вариант~\ic{else}. Вот так можно переписать предыдущий пример
с~помощью~\ic{case}:

\begin{code:lisp}
(case x
  ((flip) 'flop)
  ((flip) 'flip)
  (else (list x "neither flip nor flop")) )
\end{code:lisp}

\indexC{lambda}
\indexC{let}\indexC{let*}\indexC{letrec}
\indexC{set"!}\indexC{quote}
Функции определяются формой~\ic{lambda}. За~словом~\ic{lambda} следует список
аргументов, а после него "--- последовательность выражений, которые описывают
собственно вычисление функции. Формы~\ic{let}, \ic{let*} и~\ic{letrec}
определяют локальные переменные (они отличаются тонкостями вычисления начальных
значений определяемых переменных). Значения переменных в~дальнейшем можно
изменять с~помощью формы~\ic{set!}. Для записи литералов используется
форма~\ic{quote}.

\indexC{define}
\indexCS{define}{синтаксис}
\indexR{синтаксис!define@\protect\ic{define}}
С~помощью формы~\ic{define} можно назначить имя любому значению. У~неё есть
особые возможности, которые мы будем использовать. В~частности, возможность
использовать её как подобие \ic{let}, а также вариант синтаксиса этой формы,
позволяющий удобнее определять функции. Вот, что имеется ввиду:

\begin{code:lisp}
(define (rev l)
  (define nil '())
  (define (reverse l r)
    (if (pair? l) (reverse (cdr l) (cons (car l) r)) r))
  (reverse l nil) )
\end{code:lisp}

\noindent Без синтаксического сахара этот пример выглядит так:

\begin{code:lisp}
(define rev
  (lambda (l)
    (letrec ((reverse (lambda (l r)
                        (if (pair? l) (reverse (cdr l)
                                               (cons (car l) r))
                            r) )))
      (reverse l '()) ) ) )
\end{code:lisp}

На~этом мы заканчиваем наш краткий обзор Scheme.


\section*{Исходный код}\label{pref/sect:source}

Программы (интерпретируемые и скомпилированные), приведённые в~этой книге,
объектную систему и тесты для них можно забрать по следующему адресу:
\begin{quote}
\url{http://pagesperso-systeme.lip6.fr/Christian.Queinnec/Books/LiSP-2ndEdition-2006Dec11.tgz}
\end{quote}

Электронный адрес автора книги:
\href{mailto:Christian.Queinnec@lip6.fr}{\nolinkurl{Christian.Queinnec@lip6.fr}}


\section*{Рекомендуемая литература}\label{pref/sect:reading}

Так как подразумевается, что вы уже знаете Scheme, мы будем ссылаться на
традиционные~\cite{as85,sf89}.

Чтобы получить от книги больше, имеет смысл поглядывать в~другие руководства:
\CommonLisp~\cite{ste90}, Dylan~\cite{app92b}, \EuLisp~\cite{pe92},
\ISLisp~\cite{iso94}, \LeLisp~\cite{cdd+91}, Oaklisp~\cite{lp88},
Scheme~\cite{kcr98}, T~\cite{ram84}, Talk~\cite{ilo94}.

Наконец, для лучшего понимания языков программирования в~целом будет полезной
книга~\cite{bg94}.{\fnstyle{\RaggedRight}\trnote*{Кроме того, лично я хотел~бы
посоветовать замечательную книгу \textit{Franklyn~Turbak and David~Gifford with
Mark~A.~Sheldon.} Design Concepts in Programming Languages. "--- The MIT Press,
2008. "--- 1352~p. "--- \emph{Прим.~перев.}}}
